\documentclass[12pt]{charter}
\usepackage{wrapfig}
\usepackage{multirow}

\settitle{Project Charter}{Template for Project Charter}
\setauthor{YOUR NAME HERE}{YOUR EMAIL HERE}
\setstartdate

\begin{document}
\pagenumbering{alph}
\addToPDFBookmarks{0}{Title}{a}
\maketitle

\pagenumbering{roman}
\setcounter{page}{2}

%\renewcommand{\contentsname}{Table of Contents}
%\addToPDFBookmarks{0}{Table of Contents}{b}
%\tableofcontents

%\addToTOC{List of Figures}
%\listoffigures

%\addToTOC{List of Tables}
%\listoftables

\newpage
\pagenumbering{arabic}
\setcounter{page}{1}

%The comment text was taken from www.ProjectManagementDocs.com with the 
%exception of the Roles section
\section*{Project Description}
\label{sec:description}
%This section provides a high-level description of the project. This 
%description should not contain too much detail but should provide general 
%information about what the project is, how it will be done, and what it is 
%intended to accomplish. As the project moves forward the details will be 
%developed, but for the project charter, high-level information is what should 
%be provided.

	\subsection*{Project Objectives and Success Criteria}
	\label{subsec:objandsuccess}
	%Objectives should be SMART: Specific, Measurable, Attainable, Realistic, and
	%Time-bound. The project manager must be able to track these objectives in 
	%order to determine if the project is on the path to success. Vague, confusing,
	%and unrealistic objectives make it difficult to measure progress and success.
	
	\subsection*{Assumptions}
	%The preliminary scope statement is a general paragraph which highlights what 
	%the project will include, any high-level resource or requirement descriptions, 
	%and what will constitute completion of the project.  This preliminary scope 
	%statement is exactly that: preliminary.  All of this information will be 
	%expanded upon in greater detail as the project moves forward and undergoes 
	%progressive elaboration.

\section*{Scope}
\label{sec:scope}
%The preliminary scope statement is a general paragraph which highlights what the 
%project will include, any high-level resource or requirement descriptions, and 
%what will constitute completion of the project.  This preliminary scope statement 
%is exactly that: preliminary.  All of this information will be expanded upon in 
%greater detail as the project moves forward and undergoes progressive elaboration

\section*{Deliverables}
\label{sec:deliverables}
%This section should list all of the deliverables that the customer, project sponsor, 
%or stakeholders require upon the successful completion of the project.  Every 
%effort must be made to ensure this list includes all deliverables and project sponsor 
%approval must be required for adding additional deliverables in order to avoid scope 
%creep.

\section*{Budget and Cost}
\label{sec:budgetcost}
%The summary budget should contain general cost components and their planned costs.  
%As the project moves forward these costs may change as all tasks and requirements 
%become clearer.  Any changes must be communicated by the project manager.

\section*{Project Roles}
\label{sec:roles}
%Here we define the various people currently invovled with the project and their role.
\begin{table}[h]
  \label{tbl:roles}
  \begin{center}
    \begin{tabular}{c|c}
      Name & Role\\\hline
      YOUR NAME HERE & Project Manager\\\hline
      & Sponsor\\\hline
      & Customer / User Contact\\\hline
      & Team Members\\\hline
      & Stakeholders\\\hline
    \end{tabular}
  \end{center}
\end{table}

\section*{Summary Milestone Schedule}
\label{sec:milestones}
%This section provides an estimated schedule of all high-level project milestones.  
%It is understood that this is an estimate and will surely change as the project 
%moves forward and the tasks and milestones and their associated requirements are 
%more clearly defined.

\section*{Requirements \& Constraints}
\label{sec:require-n-constrain}
	\subsection*{Requirements}
	\label{subsec:requirements}
	%The project team should develop a list of all high-level project requirements.
	%These requirements are clear guidelines within which the project must conform 
	%and may be a result of input from the project sponsor, customer, stakeholders, 
	%or the project team. 
	
	\subsection*{Constraints}
	\label{subsec:constraints}
	%Constraints are restrictions or limitations that the project manager must deal 
	%with pertaining to people, money, time, or equipment.  It is the project 
	%manager’s role to balance these constraints with available resources in order 
	%to ensure project success.

\section*{Risks}
\label{sec:risks}
%All projects have some form of risk attached.  This section should provide a list of 
%high-level risks that the project team has determined apply to this project.

\approvalsignature

\end{document}
